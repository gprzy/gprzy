%----------------------------------%
% Este arquivo é uma adaptação de  % 
% https://github.com/sb2nov/resume %
%----------------------------------%

\documentclass[letterpaper,11pt]{article}

\usepackage{latexsym}
\usepackage[empty]{fullpage}
\usepackage{titlesec}
\usepackage{marvosym}
\usepackage[usenames,dvipsnames]{color}
\usepackage{verbatim}
\usepackage{enumitem}
\usepackage[hidelinks]{hyperref}
\usepackage{fancyhdr}
\usepackage[english]{babel}
\usepackage{tabularx}
\usepackage[dvipsnames]{xcolor}
\usepackage{emoji}
\usepackage[utf8]{inputenc}

\input{glyphtounicode}

\pagestyle{fancy}
\fancyhf{} % clear all header and footer fields
\fancyfoot{}
\renewcommand{\headrulewidth}{0pt}
\renewcommand{\footrulewidth}{0pt}

% Adjust margins
\addtolength{\oddsidemargin}{-0.5in}
\addtolength{\evensidemargin}{-0.5in}
\addtolength{\textwidth}{1in}
\addtolength{\topmargin}{-.5in}
\addtolength{\textheight}{1.0in}

\urlstyle{same}

\raggedbottom
\raggedright
\setlength{\tabcolsep}{0in}

% Sections formatting
\titleformat{\section}{
  \vspace{-4pt}\scshape\raggedright\large
}{\color{RoyalBlue}}{0em}{\color{black}}[\color{RoyalBlue}\titlerule \vspace{-5pt}]

% Ensure that generate pdf is machine readable/ATS parsable
\pdfgentounicode=1

% Custom commands
\newcommand{\resumeItem}[2]{
  \item\small{
    \textbf{#1}{#2 \vspace{-2pt}}
  }
}

% Just in case someone needs a heading that does not need to be in a list
\newcommand{\resumeHeading}[4]{
    \begin{tabular*}{0.99\textwidth}[t]{l@{\extracolsep{\fill}}r}
      \textbf{#1} & #2 \\
      \textit{\small#3} & \texttt{\small #4} \\
    \end{tabular*}\vspace{-5pt}
}

\newcommand{\resumeSubheading}[4]{
  \vspace{-1pt}\item
    \begin{tabular*}{0.97\textwidth}[t]{l@{\extracolsep{\fill}}r}
      \textbf{#1} & #2 \\
      {#3} & \small #4 \\
    \end{tabular*}\vspace{-5pt}
}

\newcommand{\resumeSubSubheading}[2]{
    \begin{tabular*}{0.97\textwidth}{l@{\extracolsep{\fill}}r}
      {#1} & \small #2 \\
    \end{tabular*}\vspace{-5pt}
}

\newcommand{\resumeSubItem}[2]{\resumeItem{#1}{#2}\vspace{-4pt}}

\renewcommand{\labelitemii}{$\circ$}

\newcommand{\resumeSubHeadingListStart}{\begin{itemize}[leftmargin=*]}
\newcommand{\resumeSubHeadingListEnd}{\end{itemize}}
\newcommand{\resumeItemListStart}{\begin{itemize}}
\newcommand{\resumeItemListEnd}{\end{itemize}\vspace{-5pt}}

\begin{document}

% Heading
\begin{tabular*}{\textwidth}{l@{\extracolsep{\fill}}r}
  \textbf{{\LARGE Gabriel Przytocki}} & \small{\textbf{E-mail}:} \href{mailto:gabrielhprzytocki@gmail.com}{\color{black}\small{\texttt{gabrielhprzytocki@gmail.com}}} \\
  \texttt{Cientista de Dados | Engenheiro de Machine Learning} & \small{\textbf{Github}}: \href{https://github.com/gprzy}{\color{black}\small{\texttt{https://github.com/gprzy}}}
\end{tabular*}

% Education
\section{Formação Acadêmica}
  \resumeSubHeadingListStart
    \resumeSubheading
      {Pontifícia Universidade Católica do Paraná}{Curitiba, Brasil}
      {\color{gray}\textit{Bacharel em Ciência da Computação}}{Fev. 2019 - Dez. 2022}
  \resumeSubHeadingListEnd

% Experience
\section{Experiência}
  \resumeSubHeadingListStart

    \resumeSubheading
      {Junto Seguros}{Curitiba, Brasil}
      {$\rightarrow$ \color{black}\texttt{Cientista de Dados}}{Jan. 2022 - o momento}
        \begin{itemize}
        \small{
            \color{gray}{
            \item {Criação de modelos de \texttt{Machine Learning};
                \hspace{60pt}  Orquestração de processos com Airflow;}
            \item {Criação de \textit{Pipelines} de Machine Learning;
                \hspace{60pt}  \textit{Computer Vision} (\texttt{OpenCV} e \texttt{Pytesseract OCR});}
            \item Aplicação de boas práticas de MLOPs;
                \hspace{77pt}  \textit{Pipelines} de extração de dados (\texttt{ETL});}
            \item Modelagem relacional de dados (\texttt{SQL Server});
                \hspace{46pt}  Python (\texttt{scikit-learn, tensorflow, numpy, pandas});}
            }
        }
        \end{itemize}
      
    % Multiple positions
    \resumeSubSubheading
    {$\rightarrow$ \color{black}\texttt{Engenheiro de Dados Pleno}}{Out. 2021 - Fev. 2022}
    \begin{itemize}
    \small{
        \color{gray}{
        \item{Processos de ETL (\textit{Extract, Transform, Load});
            \hspace{39pt}  \textit{Pipelines} de extração de dados (Airflow \& \texttt{SQL} Server);}
        \item {Modelagem relacional de dados (\texttt{T-SQL});
            \hspace{66pt}  \textit{Webscraping} de dados públicos (Selenium \& \texttt{BS5});}
        \item{Criação de fontes de dados (\texttt{SQL} Server \& Tableau);
            \hspace{19pt}  Consultas em \texttt{SQL} (\texttt{T-SQL});}
        }
    }
    \end{itemize}

    \resumeSubheading
      {Pontifícia Universidade Católica do Paraná}{Curitiba, Brasil}
      {$\rightarrow$ \color{black}\texttt{PIBIC Undergraduate Researcher}}{Jul. 2021 - Jul. 2022}
      \begin{itemize}
      \small{
          \color{gray}{
          \item {Previsão de ações em Séries Temporais;
            \hspace{74pt}  Indicadores de Análise Técnica;}
          \item {Otimização com Algoritmos Genéticos (\texttt{AG});
            \hspace{55pt}  Simulações de investimentos (\textit{Backtesting});}
          \item {Redes neurais (\texttt{CNN} e \texttt{LSTM});
            \hspace{124pt}  Modelos estatísticos (\texttt{AR}, \texttt{ARMA} \& \texttt{ARIMA});}
        }
      }
      \end{itemize}
      
      \resumeSubSubheading
      {$\rightarrow$ \color{black}\texttt{PIBIC Undergraduate Researcher}}{Ago. 2020 - Ago. 2021}
      \begin{itemize}
      \small{
          \color{gray}{
          \item {Previsão de ações em Séries Temporais;
            \hspace{74pt}  Aplicação de Redes Neurais (\texttt{CNN} e \texttt{LSTM});}
          \item {Aplicação de modelos estatísticos (Autorregressivo);
            \hspace{20pt}  Simulações de investimentos (\textit{Backtesting});}
        }
      }
      \end{itemize}

    \resumeSubheading
      {Mirum Agency}{Curitiba, Brasil}
      {$\rightarrow$ \color{black}\texttt{Engenheiro de Dados}}{Nov. 2020 - Out. 2021}
      \begin{itemize}
        \small{
            \color{gray}{
            \item {Desenvolvimento de \textit{pipelines} de extração dados;
                \hspace{34pt}  \textit{Extract, Transform} \& \textit{Load} (\texttt{ETL});}
            \item {Microsserviços (Google Cloud Platform);
                \hspace{69pt} Cloud Functions \& BigQuery;}
            \item {APIs REST\textit{ful} (Python \& Flask);
                \hspace{99pt}  Ambiente conteinerizado (Docker);}
            \item {\textit{Datawarehouse} (\texttt{SQL} \& BigQuery);
                \hspace{94pt}  \textit{Web Tracking} (Google Analytics, TagManager);}
            }
        }
        \end{itemize}
          
      % Multiple positions
      \resumeSubSubheading
      {$\rightarrow$ \color{black}\texttt{Software Developer Intern}}{Mar. 2020 - Nov. 2020}
      \begin{itemize}
      \small{
          \color{gray}{
          \item {Javascript, \texttt{HTML5, CSS3}, JQuery \& Booststrap;
            \hspace{37pt}  Implementações de \textit{e-mail marketing}};
          }
       }
       \end{itemize}

    \resumeSubheading
      {Viasoft Korp}{Curitiba, Brasil}
      {$\rightarrow$ \color{black}\texttt{Software Developer Intern}}{Jul. 2019 - Dez. 2019}
        \begin{itemize}
        \small{
            \color{gray}{
            \item {Desenvolvimento \textit{back-end} em Delphi e \texttt{SQL} Server;
                \hspace{26pt}  \texttt{MVVM} e \texttt{MVC} e \textit{framework} de testes unitários;}
            }
        }
        \end{itemize}

  \resumeSubHeadingListEnd
  
  \section{Idiomas}
  \small{
      \color{gray}{
      \textbf{Português}: Fluente ou Nativo \hspace{18pt} \textbf{Inglês}: Nível Avançado \hspace{18pt}
      }
  }
  
\newpage

% Projects
\section{Projetos}
  \resumeSubHeadingListStart
    
    \resumeSubItem{Avaliação do Desempenho de Técnicas de Deep Learning para Classificação do Comportamento Temporal de Ações: }
      {\color{gray} Relatório concluído de iniciação científica, referente a implementação e avaliação de técnicas de redes neurais (\texttt{CNN} e \texttt{LSTM}) em comparação com o modelo estatístico Autorregressivo (\texttt{AR}) na previsão em Séries Temporais Financeiras. \href{https://github.com/gprzy/PETR4.SA-forecasting}{$\rightarrow$ \color{black}\texttt{Link para GitHub}}}
    \vspace{7pt}
    
    \resumeSubItem{Previsão do Valor Futuro de Ações a Partir do Uso de Indicadores Técnicos: }
      {\color{gray} Relatório concluído de iniciação científica, referente a implementação de combinações otimizadas por algoritmos genéticos de Indicadores Técnicos na previsão de ações em Séries Temporais Financeiras, juntamente com redes \texttt{CNN} e \texttt{LSTM}.
      \href{https://github.com/gprzy/ta-forecasting}{$\rightarrow$ \color{black}\texttt{Link para GitHub}}}
    \vspace{7pt}

    \resumeSubItem{Aplicação de Fatores de Decaimento em Sistemas Adaptativos de Recomendação: }
      {\color{gray} Trabalho de Conclusão de Curso (TCC) em andamento, referente ao curso de Bacharelado em Ciência da Computação. Aplicações de métodos de esquecimento (\textit{forgetting}) de informações obsoletas nos algoritmos adaptativos (\textit{stream}) \texttt{ISGD} e \texttt{IBPRMF}.}
    \vspace{7pt}
  \resumeSubHeadingListEnd
 
 \section{Competências e Tecnologias}
 
 \resumeSubHeadingListStart
    \color{gray}
    \small{
        \item {\textbf{Machine Learning}: \texttt{scikit-learn} \& \texttt{tensorflow};}
            \hspace{65pt} \textbf{Versionamento}: \texttt{Git, GitHub};
        \item {\textbf{Estatística}: Descritiva, Inferência (\texttt{scipy});}
            \hspace{104pt} \textbf{Conteneirização}: \texttt{Docker, docker-compose};
        \item{
         {{\textbf{Banco de Dados}: \texttt{SQL (T-SQL)} \& \texttt{SQL} Server;}
            \hspace{90pt}\textbf{Linguagens}: \texttt{Python, C++, R, Julia \& Java};}
        }
        \item{
        {{\textbf{Processamento Distribuído}: \texttt{pyspark} \& \texttt{Hadoop MapReduce};
            \hspace{15pt} \textbf{Soft Skills}: \textit{Problem Solving, Communication};}}
        \item{\textbf{Microsserviços e} \texttt{APIs} \texttt{REST}\textit{ful}: {Cloud Functions, \texttt{flask}};
            \hspace{38pt} \textbf{Cloud}: GCP \& AWS};
        }
        \item{ \textbf{ETL}: Airflow, \texttt{SQL} Server \& Cloud Functions;
            \hspace{93pt} \textbf{Manipulação de Dados}: \texttt{pandas} \& \texttt{numpy};}
        \item{\textbf{Visualização de Dados}: \texttt{matplotlib} \& \texttt{seaborn};
            \hspace{70pt} \textbf{Séries Temporais}: \texttt{statsmodels} \& \texttt{pycaret};}
        \item{\textbf{MLOPs}: \texttt{mlflow, metaflow};
            \hspace{163pt} \textbf{NLP}: \texttt{spacy} \& \texttt{nltk};}
    }
 \resumeSubHeadingListEnd
 
 \section{Formação Complementar}
 \resumeSubHeadingListStart
    \color{gray}
    \small{
        \item {\textit{Machine Learning Specialization} - (\textbf{Stanford, Coursera});}
        \item {\textit{Deep Learning Specialization} - (\textbf{DeepLearning.AI, Coursera});}
        \item {\textit{Architecting with Google Compute Engine} (\textbf{Coursera});}
        \item {\textit{Elastic Google Cloud Infrastructure: Scaling and Automation} (\textbf{Coursera});}
        \item {\textit{Essential Google Cloud Infrastructure: Core Services} (\textbf{Coursera});}
        \item {\textit{Essential Google Cloud Infrastructure: Foundation} (\textbf{Coursera});}
        \item {\textit{Google Cloud Platform Fundamentals: Core Infrastructure} (\textbf{Coursera});}
        \item {\textit{Reliable Google Cloud Infrastructure: Design and Process} (\textbf{Coursera});}
        \item {\textit{Analyst Developer} (Certificado Profissional - \textbf{Pontifícia Universidade Católica do Paraná});}
        \item {\textit{Data Scientist} (Certificado Profissional - \textbf{Pontifícia Universidade Católica do Paraná});}
    }
 \resumeSubHeadingListEnd

\end{document}