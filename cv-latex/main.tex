%----------------------------------%
% Este arquivo é uma adaptação de  % 
% https://github.com/sb2nov/resume %
%----------------------------------%

\documentclass[letterpaper,11pt]{article}

\usepackage{latexsym}
\usepackage[empty]{fullpage}
\usepackage{titlesec}
\usepackage{marvosym}
\usepackage[usenames,dvipsnames]{color}
\usepackage{verbatim}
\usepackage{enumitem}
\usepackage[hidelinks]{hyperref}
\usepackage{fancyhdr}
\usepackage[english]{babel}
\usepackage{tabularx}
\usepackage[dvipsnames]{xcolor}
\usepackage{emoji}

\input{glyphtounicode}

\pagestyle{fancy}
\fancyhf{} % clear all header and footer fields
\fancyfoot{}
\renewcommand{\headrulewidth}{0pt}
\renewcommand{\footrulewidth}{0pt}

% Adjust margins
\addtolength{\oddsidemargin}{-0.5in}
\addtolength{\evensidemargin}{-0.5in}
\addtolength{\textwidth}{1in}
\addtolength{\topmargin}{-.5in}
\addtolength{\textheight}{1.0in}

\urlstyle{same}

\raggedbottom
\raggedright
\setlength{\tabcolsep}{0in}

% Sections formatting
\titleformat{\section}{
  \vspace{-4pt}\scshape\raggedright\large
}{\color{RoyalBlue}}{0em}{\color{black}}[\color{RoyalBlue}\titlerule \vspace{-5pt}]

% Ensure that generate pdf is machine readable/ATS parsable
\pdfgentounicode=1

% Custom commands
\newcommand{\resumeItem}[2]{
  \item\small{
    \textbf{#1}{#2 \vspace{-2pt}}
  }
}

% Just in case someone needs a heading that does not need to be in a list
\newcommand{\resumeHeading}[4]{
    \begin{tabular*}{0.99\textwidth}[t]{l@{\extracolsep{\fill}}r}
      \textbf{#1} & #2 \\
      \textit{\small#3} & \texttt{\small #4} \\
    \end{tabular*}\vspace{-5pt}
}

\newcommand{\resumeSubheading}[4]{
  \vspace{-1pt}\item
    \begin{tabular*}{0.97\textwidth}[t]{l@{\extracolsep{\fill}}r}
      \textbf{#1} & #2 \\
      {#3} & \textit{\small #4} \\
    \end{tabular*}\vspace{-5pt}
}

\newcommand{\resumeSubSubheading}[2]{
    \begin{tabular*}{0.97\textwidth}{l@{\extracolsep{\fill}}r}
      {#1} & \textit{\small #2} \\
    \end{tabular*}\vspace{-5pt}
}

\newcommand{\resumeSubItem}[2]{\resumeItem{#1}{#2}\vspace{-4pt}}

\renewcommand{\labelitemii}{$\circ$}

\newcommand{\resumeSubHeadingListStart}{\begin{itemize}[leftmargin=*]}
\newcommand{\resumeSubHeadingListEnd}{\end{itemize}}
\newcommand{\resumeItemListStart}{\begin{itemize}}
\newcommand{\resumeItemListEnd}{\end{itemize}\vspace{-5pt}}

\begin{document}

% Heading
\begin{tabular*}{\textwidth}{l@{\extracolsep{\fill}}r}
  \textbf{{\LARGE Gabriel Przytocki}} & \small{E-mail:} \href{mailto:gabrielhprzytocki@gmail.com}{\color{blue}\small{\texttt{gabrielhprzytocki@gmail.com}}} \\
  \texttt{Data Scientist} & \small{Github}: \href{https://github.com/gprzy}{\color{blue}\small{\texttt{https://github.com/gprzy}}}
\end{tabular*}

% Education
\section{Formação Acadêmica}
  \resumeSubHeadingListStart
    \resumeSubheading
      {Pontifícia Universidade Católica do Paraná}{Curitiba, Brasil}
      {\color{gray}\textit{Bacharel em Ciência da Computação}}{Fev. 2019 - Dez. 2022}
  \resumeSubHeadingListEnd

% Experience
\section{Experiência}
  \resumeSubHeadingListStart

    \resumeSubheading
      {Junto Seguros}{Curitiba, Brasil}
      {$\rightarrow$ \color{WildStrawberry}\texttt{Data Scientist}}{Jan. 2022 - o momento}
      \resumeItemListStart
        \resumeItem{}
          {\color{gray}Criação de modelos resilientes de \texttt{NLP} para mineração de texto em imagens, utilizando \texttt{Python}, \texttt{OpenCV}, \texttt{Tesseract OCR}, \texttt{regex} e heurísticas específicas de extração.}
      \resumeItemListEnd
      
    % Multiple positions
    \resumeSubSubheading
    {$\rightarrow$ \color{WildStrawberry}\texttt{Engenheiro de Dados Pleno}}{Out. 2021 - Fev. 2022}
    \resumeItemListStart
        \resumeItem{}
         {\color{gray}Desenvolvimento de processos de extração, transformação e carregamento de dados (\texttt{ETL}) com \texttt{SQL} Server, \texttt{Python} e \texttt{Airflow}; \textit{webscraping}, consultas, validações e alterações via \texttt{SQL} (\texttt{T-SQL}), bem como otimização e manutenção de fontes publicadas a serem utilizadas pelo time de BI para visualização de dados, com Tableau.}
    \resumeItemListEnd

    \resumeSubheading
      {Pontifícia Universidade Católica do Paraná}{Curitiba, Brasil}
      {$\rightarrow$ \color{WildStrawberry}\texttt{PIBIC Researcher}}{Ago. 2020 - o momento}
      \resumeItemListStart
        \resumeItem{}
         {\color{gray}Projetos de pesquisa em iniciação científica, envolvendo aplicações de aprendizagem de máquina no comportamento temporal de ações. Possui um projeto já concluído onde utiliza redes neurais (\texttt{CNN} e \texttt{LSTM}) em comparação com modelos estatísticos, bem como outro em andamento, onde utiliza indicadores de análise técnica.}
    \resumeItemListEnd

    \resumeSubheading
      {Mirum Agency}{Curitiba, Brasil}
      {$\rightarrow$ \color{WildStrawberry}\texttt{Data Engineer Intern}}{Nov. 2020 - Out. 2021}
      \resumeItemListStart
        \resumeItem{}
          {\color{gray}Auxiliar na manutenção, configuração, documentação e melhoria de procedimentos referentes a processamento de dados, \texttt{ETLs} e afins, atuando em ambiente \textit{cloud} (\textit{Google Cloud Platform}). Automatizações, desenvolvimento e integração de \texttt{APIs} utilizando \texttt{Python} em Microsserviços (\texttt{Cloud Functions}); ambiente conteinerizado (\texttt{Docker}); solução robusta de dados com datawarehouse; consultas e validações via \texttt{SQL}. Trabalho em metodologia ágil (\textit{Scrum}).}
      \resumeItemListEnd
          
      % Multiple positions
      \resumeSubSubheading
      {$\rightarrow$ \color{WildStrawberry}\texttt{Software Developer Intern}}{Mar. 2020 - Nov. 2020}
      \resumeItemListStart
        \resumeItem{}
         {\color{gray}Desenvolvimentos Front-End em \texttt{HTML5}, \texttt{CSS3}, \texttt{Javascript}, \texttt{JQuery} e \texttt{Booststrap}. Implementação de layouts, manutenção e otimização de pages em sites e campanhas de \textit{e-mail marketing}.}
     \resumeItemListEnd

    \resumeSubheading
      {Viasoft Korp}{Curitiba, Brasil}
      {$\rightarrow$ \color{WildStrawberry}\texttt{Software Developer Intern}}{Jul. 2019 - Dez. 2019}
      \resumeItemListStart
        \resumeItem{}
          {\color{gray}Auxiliar no desenvolvimento de módulos do sistema \texttt{ERP}; utilização de \texttt{Delphi} e \texttt{SQL} Server aplicado em projetos Back-End, estruturados em \texttt{MVVM} e \texttt{MVC}. Aplicação de Framework de Testes Unitários e metodologia ágil (\textit{Scrum}).}
      \resumeItemListEnd

  \resumeSubHeadingListEnd

% Projects
\section{Projetos}
  \resumeSubHeadingListStart
    
    \resumeSubItem{Avaliação do Desempenho de Técnicas de Deep Learning para Classificação do Comportamento Temporal de Ações: }
      {\color{gray}Implementação e avaliação de técnicas de redes neurais (\texttt{CNN} e \texttt{LSTM}) em comparação com o modelo estatístico Autorregressivo (\texttt{AR}) na previsão em Séries Temporais Financeiras. Algoritmos de Investimento e simulações de operação com as ações \texttt{PETR4} (Petrobrás). \href{https://github.com/gprzy/PETR4.SA-forecasting}{$\rightarrow$ \color{blue}\texttt{Link para GitHub}}}
    \vspace{7pt}
    \resumeSubItem{Previsão do Valor Futuro de Ações a Partir do Uso de Indicadores Técnicos: }
      {\color{gray}Implementação de combinações otimizadas por algoritmos genéticos de Indicadores Técnicos na previsão de ações em Séries Temporais Financeiras. Backtesting das técnicas utilizadas.}
  \resumeSubHeadingListEnd

% Skills
\section{Competências}
 \resumeSubHeadingListStart
    \item{
     \textbf{Linguagens: }{\color{gray}\texttt{Python, C++, SQL, Java, Bash}}
     \hfill
     \textbf{Tecnologias: }{\color{gray}\texttt{Docker, GCP, VScode, Git, Linux}}
    }
    \item{
    \textbf{Soft Skills: }{\color{gray}\textit{Problem Solving, Communication}}
    \hfill
    \textbf{Tópicos: }{\color{gray}\textit{Machine Learning, Statistics, Time Series}}
    }
 \resumeSubHeadingListEnd

\end{document}